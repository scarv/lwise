% Copyright (C) 2019 SCARV project <info at scarv.org>
%
% Use of this source code is restricted per the MIT license, a copy of which 
% can be found at https://opensource.org/licenses/MIT (or should be included 
% as LICENSE.txt within the associated archive or repository).

\documentclass[9pt]{beamer}

\usepackage{slide}

\begin{document}

% =============================================================================

\begin{frame}
\titlepage

\centerline{(disclaimer: some of the above are involved in~\cite{NIST:LWC:sparkle})}
\end{frame}

% =============================================================================

\begin{frame}

\begin{block}{Definition}
NIST define~\cite[Section 3.4]{NIST:LWC:call}
\[
\mbox{lightweight cryptography ~~$\simeq$~~ ``tailored for resource-constrained devices''}
\]
e.g.,

\begin{itemize}
\item efficient on constrained hard/software platforms (vs. existing standards),
\item efficient for short messages,
\item ameanable to countermeasures against implementation attacks,
\item ...
\end{itemize}

\noindent
with ``efficient'' read as low-latency, low-footprint, low-power, etc.
\end{block}

\end{frame}

\begin{frame}

%$56$ round-$1$ candidates
%$32$ round-$2$ candidates
%$10$           finalists

\begin{center}
\begin{small}
\begin{tabular}{|l@{\;}c|cc|p{4cm}|}
\hline
Name              & Specification            & AEAD       & Hash       & Component(s)                                                                                                              \\
\hline
\hline
Grain128-AEAD     & \cite{NIST:LWC:grain}    & \checkmark &            & \only<1>{\adjustbox{right}{            Stream cipher}}\only<2>{L/NFSRs                                                  } \\
\hline
GIFT-COFB         & \cite{NIST:LWC:gift}     & \checkmark &            & \only<1>{\adjustbox{right}{            Block  cipher}}\only<2>{GIFT-128                                                 } \\
Romulus           & \cite{NIST:LWC:romulus}  & \checkmark & \checkmark & \only<1>{\adjustbox{right}{(Tweakable) Block  cipher}}\only<2>{Skinny-128-384+                                          } \\
\hline
Ascon             & \cite{NIST:LWC:ascon}    & \checkmark & \checkmark & \only<1>{\adjustbox{right}{              Permutation}}\only<2>{Ascon-$p$                                                } \\
Elephant          & \cite{NIST:LWC:elephant} & \checkmark &            & \only<1>{\adjustbox{right}{              Permutation}}\only<2>{Spongent-$\pi[n]$ or Keccak-$f[m]$                       } \\
PHOTON-Beetle     & \cite{NIST:LWC:photon}   & \checkmark & \checkmark & \only<1>{\adjustbox{right}{              Permutation}}\only<2>{${\tt PHOTON}_{256}$                                     } \\
Schwaemm and Esch & \cite{NIST:LWC:sparkle}  & \checkmark & \checkmark & \only<1>{\adjustbox{right}{              Permutation}}\only<2>{Sparkle                            (inc. Alzette ARX-box)} \\
Xoodyak           & \cite{NIST:LWC:xoodyak}  & \checkmark & \checkmark & \only<1>{\adjustbox{right}{              Permutation}}\only<2>{Xoodoo                                                   } \\
ISAP              & \cite{NIST:LWC:isap}     & \checkmark &            & \only<1>{\adjustbox{right}{              Permutation}}\only<2>{Ascon-$p$         or Keccak-$f[m]$                       } \\
TinyJAMBU         & \cite{NIST:LWC:jambu}    & \checkmark &            & \only<1>{\adjustbox{right}{(Keyed)       Permutation}}\only<2>{$P_n$                              (inc. LFSR)           } \\
\hline
\end{tabular}
\end{small}
\end{center}

\end{frame}

% -----------------------------------------------------------------------------

% scope
% - consider RV32 {\em and} RV64, so focus on Rocket-based hardware
% - ignore hash API
% - ignore ISAP (since this uses Ascon-$p$ or Keccak-$f[m]$ permutation)
% - use ``partial'' component (e.g., only encryption) where possible

% strategy
% - for each candidate
%   - analysis, then on-paper ISE design
%   - software implementation using stock tool-chain plus {\tt .insn}
%   - simulate using (patched) Spike
%   - hardware implementation using Rocket

% design criteria ~= CFU specification (https://cfu.readthedocs.io/en/latest)
% - strictly $3$-address (i.e., $2$-input, $1$-output) instructions
% - minimise additional state (e.g., CSRs)

% special-purpose
% - there are some cases where an interface change helps a lot (e.g., Photon initialises a state matrix row-wise but the natural representation/implementation is column-wise)
% - there are some cases where a 32-bit ISE doesn't ``scale'' naturally to 64-bit (e.g., the AES-like approach for Photon)

% general-purpose
% - several shift/rotate + X for X in \SET{ XOR, ... }
% - SWAPMOVE is a generic primitive used in several implementations based on fix-slicing
% - there are a few cases where ``small'' LFSRs are involved

% =============================================================================

\begin{frame}[allowframebreaks]{References}{}

\nocite{*} \printbibliography[heading=none]

\end{frame}

% =============================================================================

\end{document}
